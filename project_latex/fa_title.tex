%% -!TEX root = AUTthesis.tex
% در این فایل، عنوان پایان‌نامه، مشخصات خود، متن تقدیمی‌، ستایش، سپاس‌گزاری و چکیده پایان‌نامه را به فارسی، وارد کنید.
% توجه داشته باشید که جدول حاوی مشخصات پروژه/پایان‌نامه/رساله و همچنین، مشخصات داخل آن، به طور خودکار، درج می‌شود.
%%%%%%%%%%%%%%%%%%%%%%%%%%%%%%%%%%%%
% دانشکده، آموزشکده و یا پژوهشکده  خود را وارد کنید
\faculty{دانشکده مهندسی کامپیوتر}
% گرایش و گروه آموزشی خود را وارد کنید
\department{گرایش هوش مصنوعی و رباتیکز}
% عنوان پایان‌نامه را وارد کنید
\fatitle{شناسایی اعمال با روش یادگیری مستمر}
% نام استاد(ان) راهنما را وارد کنید
\firstsupervisor{جناب آقای دکتر محمد رحمتی}
%\secondsupervisor{استاد راهنمای دوم}
% نام استاد(دان) مشاور را وارد کنید. چنانچه استاد مشاور ندارید، دستور پایین را غیرفعال کنید.
%\firstadvisor{نام کامل استاد مشاور}
%\secondadvisor{استاد مشاور دوم}
% نام نویسنده را وارد کنید
\name{پریسا}
% نام خانوادگی نویسنده را وارد کنید
\surname{ملاحسینی}
%%%%%%%%%%%%%%%%%%%%%%%%%%%%%%%%%%
\thesisdate{مرداد 1404}

% چکیده پایان‌نامه را وارد کنید
\fa-abstract{
در سال‌های اخیر، یادگیری ماشین و به‌ویژه شبکه‌های عصبی عمیق پیشرفت چشمگیری را تجربه کرده‌اند و توانسته‌اند در حوزه‌هایی همچون بینایی ماشین، پردازش زبان طبیعی و تشخیص گفتار، به سطح عملکردی نزدیک به انسان دست یابند. دستیابی به چنین عملکردی معمولاً نیازمند پیش‌آموزش این شبکه‌ها بر روی مجموعه‌داده‌های بزرگ است، فرآیندی که امکان بهره‌برداری مجدد از مدل‌های آموزش‌دیده را در وظایف گوناگون فراهم می‌سازد.
بنابراین، در بسیاری از کاربردهای واقعی، داده‌ها به‌صورت تدریجی و در قالب وظایف متوالی در دسترس قرار می‌گیرند که لزوم استفاده از رویکردهای یادگیری پیوسته را پررنگ می‌سازد. یکی از چالش‌های اساسی این حوزه، فراموشی فاجعه‌بار است که موجب افت شدید عملکرد مدل روی وظایف گذشته، پس از یادگیری وظایف جدید، می‌شود. اخیرا با وجود توسعه‌ی روش‌های مبتنی‌بر مدل‌های زبانی بزرگ و مدل‌های بینایی-زبان، محدودیت‌هایی مانند مصرف بالای حافظه همچنان پابرجاست.
در این پژوهش، روشی با عنوان \lr{ProActionCLIP} برای یادگیری پیوسته در داده‌های ویدیویی ارائه شده است که ترکیبی از قابلیت‌های مدل بینایی-زبان \lr{Open-VCLIP} در استخراج ویژگی‌های ویدیو و سازوکار پرامپت‌های یادگیرنده در مدل \lr{L2P} را به‌کار می‌گیرد. این ترکیب، بدون نیاز به تغییر پارامترهای اصلی مدل پایه‌ی \lr{Open-VCLIP}، امکان انطباق با وظایف متوالی را فراهم کرده و با بهینه‌سازی انتخاب پرامپت‌ها، از فراموشی فاجعه‌بار جلوگیری می‌کند. نتایج آزمایش‌ها نشان می‌دهد که روش پیشنهادی علاوه بر حفظ دانش پیشین و یادگیری مؤثر دانش جدید، از نظر مصرف حافظه و منابع محاسباتی نیز کارایی بالایی دارد و می‌تواند به‌عنوان راهکاری مؤثر برای یادگیری پیوسته در حوزه‌ی تشخیص حرکت انسان در ویدیو مورد استفاده قرار گیرد.
}


% کلمات کلیدی پایان‌نامه را وارد کنید
\keywords{یادگیری پیوسته، مدل بینایی-زبان، یادگیری پرامپت، فراموشی فاجعه‌بار}



\AUTtitle
%%%%%%%%%%%%%%%%%%%%%%%%%%%%%%%%%%
\vspace*{7cm}
\thispagestyle{empty}
\begin{center}
\includegraphics[height=5cm,width=12cm]{Images/besm.jpg}
\end{center}