\chapter{جمع‌بندي و نتيجه‌گيري و پیشنهادات}
\label{chap5:concolusion}
%%%%%%%%%%%%%%%%%%%%%%%%%%%%%%%%%%%%%%%%%%%
در این پایان نامه به معرفی مدل پیشنهادی \lr{ProActionCLIP} برای یادگیری پیوسته در داده‌های ویدیویی پرداخته شد. این روش با ترکیب توانایی‌های مدل \lr{Open-VCLIP} در استخراج ویژگی‌های ویدیو و سازوکار پرامپت‌های یادگیرنده در \lr{L2P}، توانسته است بدون تغییر مستقیم پارامترهای مدل پایه‌ی \lr{Open-VCLIP}، به یادگیری پیوسته‌ی وظایف در حوزه‌ی ویدیو بپردازد. به عبارت دیگر، \lr{ProActionCLIP} با بهره‌گیری از مزیت‌های هر دو مدل مرجع، راهکاری مؤثر در یادگیری پیوسته، برای تشخیص حرکت انسان در حوزه‌ی ویدیو ارائه می‌دهد. نتایج حاصل نشان داد که این مدل، ضمن حفظ دانش پیشین و کاهش قابل‌توجه فراموشی فاجعه‌بار، از لحاظ مصرف حافظه و منابع محاسباتی عملکرد بهتری نسبت به سایر روش‌های مشابه داشته است. 
\section{پیشنهادات}
با توجه به نتایج و تحلیل‌های ارائه‌شده، چند مسیر پژوهشی برای بهبود روش پیشنهادی قابل بررسی است:

\begin{enumerate}
	\item \textbf{کاهش شباهت کلیدها بین کلاس‌های مشابه:} 
	یکی از چالش‌های اصلی در افزایش تعداد وظایف، شباهت میان برچسب‌ها است که منجر به شباهت کلیدهای متناظر آن‌ها می‌شود. این امر می‌تواند باعث انتخاب نادرست پرامپت‌ها گردد. پیشنهادی که می‌توان مطرح کرد، در نظر گرفتن یک مؤلفه در تابع زیان است که فاصله‌ی کلیدهای مربوط به برچسب‌های متفاوت را افزایش دهد. به این ترتیب، احتمال شباهت کلیدها بین کلاس‌های نزدیک کاهش یافته و دقت انتخاب پرامپت‌ها بهبود می‌یابد.
	
	\item \textbf{افزایش تعداد پرامپت‌های اختصاصی برای هر کلاس:} 
	اختصاص تعداد بیشتری پرامپت به هر کلاس می‌تواند انعطاف‌پذیری مدل را در یادگیری ویژگی‌های متنوع آن کلاس افزایش دهد و عملکرد کلی سیستم را بهبود بخشد.
\end{enumerate}