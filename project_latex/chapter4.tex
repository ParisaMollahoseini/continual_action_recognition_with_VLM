\chapter{مشخصات یک پایان نامه و گزارش علمی}
\section{مقدمه}
در این فصل، عملکرد مدل \lr{ProActionCLIP} در زمینه‌ی یادگیری پیوسته‌ی تشخیص حرکت انسان مورد ارزیابی قرار می‌گیرد. هدف از این ارزیابی، بررسی میزان صحت مدل در یادگیری وظایف جدید، ارزیابی میزان فراموشی دانش پیشین و تحلیل بهره‌وری محاسباتی آن از نظر مصرف سخت‌افزار و تعداد پارامترهای قابل‌آموزش است. به منظور ارزیابی جامع، مدل پیشنهادی با روش‌های مطرح در این حوزه مانند \lr{PIVOT} \cite{pivot}و سایر رویکردهای مرجع مقایسه می‌شود. برای این منظور، از مجموعه‌داده‌های نظیر \lr{UCF101} و سایر داده‌های ویدیویی مرسوم استفاده شده است.

در ادامه‌ی‌ این فصل، ابتدا معیارهای ارزیابی شامل صحت، میزان فراموشی معرفی می‌شوند. سپس، جزئیات مربوط به تنظیمات آزمایشی، روش‌های مقایسه‌ای و تحلیل نتایج ارائه خواهد شد تا ارزیابی مدل پیشنهادی به‌طور کامل و شفاف صورت گیرد. در اخر نیز پیچیدگی محاسباتی بررسی خواهد شد. 
% سخت افزار رو تو همون پیچیدگی محاسباتی بگم. پارامتر. شامل حافظه . بگم رو جی پی یو فلان اموزش دادم. 

\section{معیار‌های ارزیابی}
در یادگیری پیوسته، دو معیار ارزیابی اهمیت دارند. یکی از آن‌ها صحت وظایف با وجود یادگیری سایر وظایف بوده و دیگری میزان فراموشی مدل پس از یادگیری هر وظیفه است که در ادامه هر یک شرح داده خواهد شد. 
 

\section{میانگین صحت}

% دقت Top-1
\begin{equation}
	A_{top1}(t) = \frac{1}{t+1} \sum_{i=0}^{t} \text{\lr{ACC}}^{top1}_{i,t}
\end{equation}

% دقت Top-5
\begin{equation}
	A_{top5}(t) = \frac{1}{t+1} \sum_{i=0}^{t} \text{\lr{ACC}}^{top5}_{i,t}
\end{equation}


\section{میزان فراموشی}
در یادگیری پیوسته، مدل باید بتواند وظایف جدید را یاد بگیرد بدون اینکه دانش وظایف قبلی را فراموش کند. اما معمولاً پدیده‌ی فراموشی فاجعه‌آمیز رخ می‌دهد؛ یعنی مدل پس از یادگیری وظایف جدید، صحت آن روی وظایف قدیمی کاهش پیدا می‌کند. معیار فراموشی برای اندازه‌گیری میزان این افت عملکرد تعریف می‌شود و به صورت میانگین کاهش دقت در وظایف قبلی است. بر اساس \eqref{eq:forget_task_i}، فراموشی مدل روی وظیفه‌ی $i$ پس از یادگیری وظیفه‌ی $t$ بدست می‌آید. ماتریس \lr{ACC}، شامل صحت مدل روی هر وظیفه پس از یادگیری همان وظیفه و وظیفه‌های دیگر است. به دنبال آن، $\text{\lr{ACC}}_{i,k}$ نشان‌دهنده‌ی صحت مدل روی وظیفه‌ی $i$ پس از یادگیری وظیفه‌ی $k$ است. به این ترتیب، اختلاف بین بیشترین صحتی که وظیفه‌ی $i$ پس از یادگیری وظایف مختلف بدست آورده و صحتی که پس از وظیفه‌ی $t$ (آخرین وظیفه‌ی یادگرفته شده) بدست آمده، در $f_i(t)$ قرار می‌گیرد. در نهایت فراموشی برای هر وظیفه محاسبه شده و میانگین آن‌ها به عنوان فراموشی مدل پس از یادگیری وظیفه‌ی $t$، در نظر گرفته می‌شود (\eqref{eq:forget_final}). 
\begin{equation}\label{eq:forget_task_i}
	f_i(t) = \max_{k \le t} \text{\lr{ACC}}_{i,k} - \text{\lr{ACC}}_{i,t}
\end{equation}
\begin{equation}\label{eq:forget_final}
	F(t) = \frac{1}{t} \sum_{i=1}^{t} f_i(t) 
\end{equation}














